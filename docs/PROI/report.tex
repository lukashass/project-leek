\documentclass[10pt, a4paper, draft]{article}
 \usepackage[ngerman]{babel}
 \usepackage[utf8]{inputenc}
 \usepackage[T1]{fontenc}
 \usepackage{color}
 \usepackage{hyperref}

\title{Projektbreicht Smart Music Player}
\author{Anton Bracke\\Jan Eberlein\\Tom Calvin Haak\\Julian Hahn\\Nick Loewecke}

\begin{document}
\maketitle
\tableofcontents

\section{Einleitung}
\subsection{Unternehmen}
\colorbox{red}{was macht macio aus}
\subsection{Projektidee}
Im Rahmen des Projekt Informatik möchte Macio ihr Portfolio im IoT-Bereich erweitern, sowie ihren Empfangsraum im Standort Kiel verschönern.
Hierfür soll eine smarte Spielzeug-Box gebaut werden.
Smarte Spielzeuge gibt es im kommerziellen Bereich viele, daher soll dieses Projekt eine Open-Source-Alternative schaffen.\\
Genauer handelt es sich um eine Musik-Box, die NFC-Chips lesen und Spotify Connect unterstützen soll.
Auf die Box können dann Spielzeuge (z.B. in Form von kleinen Figuren) mit integrierten NFC-Chips gestellt werden, um spezifische Musik abspielen zu lassen.
Die Musik wird von Spotify-Connect über eine bereits bestehende Musik-Anlage abgespielt.
Falls es im Rahmen des Projektes möglich ist, sollen die Nutzer in der Lage sein, zwischen verschiedenen Musikanbietern zu wechseln.
NFC-Chips und die zugehörige Musik sollen über ein Web-basierte Benutzeroberfläche konfiguriert werden können.
Diese Benutzeroberfläche soll von der Box ausgeliefert und primär für Smartphone-Bedienung gestaltet werden.
Da es sich um ein Open Source Projekt mit entsprechender Lizenz handelt, muss auch eine aussagekräftige, öffentliche Dokumentation verfasst werden.
Macio stellt die benötigte Hardware zur Verfügung und unterstützt bei technischen Fragen.

\subsubsection{Minimal Requirements}
\begin{enumerate}
  \item NFC-Tags lesen, schreiben und entschlüsseln
  \item Mit Spotify Connect verbinden und arbeiten
  \item Responsive UI konzeptionieren und umsetzen
  \item Aussagekräftige Dokumentation mit Benutzerhandbuch
\end{enumerate}
\subsubsection{Stretch Goals}
\begin{enumerate}
  \item Sound Wiedergabe auf der Box selbst
  \item Unterstützung anderer Musikdienste / Plugin-Subsystem
  \item 3D-Modellierung und Print einer passenden Box
  \item Cloud-Anbindung der Box, Auslieferung des UI aus der Cloud
\end{enumerate}

\section{Machbarkeitsstudie}

\subsection{NFC Tag}
\subsubsection{lesen}
Um mit NFC Tags arbeiten zu können, müssen diese auch auf entschlüsselt bzw. gelesen werden können.
Hierfür ist ein Hardware NFC-Reader notwendig, der per USB mit der Box verbunden wird.
\colorbox{yellow}{welche library wird genutzt?}

\subsubsection{schreiben}
Einen NFC-Tag zu beschreiben ist davon abhängig, ob der NFC-Tag über einen eigenen Speicher verfügt.
Ist dieser vorhanden, können dort z.b. Kontaktdaten hinterlegt werden. Werden diese dann von einem Smartphone gelesen, öffnet sich die Kontakte-App und der auf dem NFC-Tag gespeicherte Kontakt kann abgespeichert werden.
\\
\colorbox{yellow}{Möglich, kann die gleiche library wie zum lesen genutzt werden?}

\subsection{Raspberri Pi}
\subsubsection{Docker Integration}
Docker auf einem Raspberri Pi laufen zu lassen ist möglich und in vielen Anleitungen online nachlesbar.\footnote{https://phoenixnap.com/kb/docker-on-raspberry-pi}

\subsubsection{Öffentlich zugängliches Web Interface}
Um den Web Server vom Pi von außerhalb des Netzwerkes erreichen zu können, muss innerhalb des Routers der Port 80 port forwarded werden.
Dann kann der Pi unter dessen IP xxx.xxx.xxx.xxx:80 erreicht werden.
Wenn die IP Adresse sich nicht ändern soll, kann auch eine statische IP festgelegt werden \footnote{https://howchoo.com/pi/configure-static-ip-address-raspberry-pi}. Dazu könnte dann eine Domain gekauft werden, die dann z.b. mit Cloudflare auf diese statische IP weiterleitet.
Alternativ ginge es auch ohne Port-Forwarding mit nginx und ngrok \footnote{https://vatsalyagoel.com/setting-up-a-public-web-server-using-a-raspberry-pi-3/}.

\subsubsection{URL für UI festlegen}
Über ein DNS Service, der auf dem Raspberri Pi läuft, wäre es möglich für die statische Public-IP eine eigene URL anzulegen.
Dafür sind verschiedene DNS Services möglich, potentiell ist auch eine Domain notwendig.

\subsection{User Interface}
\subsubsection{Zugriff auf NFC Reader im Cloud Szenario}
Tendenziell sollte es nicht notwendig sein, dass der Raspberri Pi, der das Web-UI hosted, auf den jeweiligen NFC-Reader der Box zugreifen muss.
Sollte das doch der Fall sein, könnte der Chip der Box ein API Call an den Raspberri Pi mit den NFC-Tag Daten machen, um das jeweilige hinterlegte Event zu triggern.

\subsubsection{Login via Spotify, Youtube, etc.}
Das Feathers Plugin bietet schon die Möglichkeit, OAuth zu nutzen, um sich über andere Services wie Spotify anzumelden.

\subsubsection{Gleichen Nutzer bei verschiedenen Loginvarianten wiedererkennen}
Um gleiche Nutzer zu erkennen, müssten Merkmale angelegt werden, über diese gleiche Nutzer wiedererkennbar wären.
Die E-Mail wäre hierbei das einzigartigste Merkmal. Über Scopes in der OAuth Anfrage kann diese auch mitgeliefert werden.
Das setzt aber voraus, dass Nutzer immer die gleiche E-Mail nutzen, was nicht immer der Fall ist.
Man könnte dem (bereits eingeloggten) User die Möglichkeit geben, weitere Accounts zu diesem hinzuzufügen, sodass diese Accounts den gleichen Nutzer einloggen.
Bei diesem Ansatz sollten dann die jeweiligen verbundenen Accounts zu einer unique User-ID verbunden werden.

\subsubsection{Musik Artwork laden}
Sollte kein Problem sein, da alle Anfragen ein Array von Bildern enthalten \footnote{https://stackoverflow.com/questions/10123804/retrieve-cover-artwork-using-spotify-api}

\subsubsection{Eigene Bilder hochladen}
Eigene Bilder hochzuladen sollte möglich sein. In unserem Kontext mit Vue.js und Node.js würde das Plugin \textit{vue-picture-input} helfen.
Damit könnte mit einem Axios Post an das Backend das Bild gesendet werden. \footnote{https://www.digitalocean.com/community/tutorials/vuejs-uploading-vue-picture-input}

\subsubsection{Spotify connect Lautsprecher auswählen}
Das Auswählen von einem spezifischen Spotify Connect Lautsprecher ist möglich.
Über einen API Call an die Spotify API mit dem Endpunkt \textit{/v1""/me""/player""/device} liefert eine Liste von allen verbundenen Geräten.
Es kann auch der letzte aktive Lautsprecher genutzt werden. Dieser hat bei \textit{is\_active} den Wert \textit{true}. \footnote{https://developer.spotify.com/documentation/web-api/guides/using-connect-web-api/}

\subsubsection{Spotify connect Lautsprecher speichern}
Die Liste von verbundenen Geräten, die über den API Call an die Spotify API erhalten wird, enthält auch einen unique Eintrag \textit{id}. Dieser kann gespeichert werden.

\subsubsection{In Cloud UI die jeweilige Box auswählen}
In dem Szenario, dass wir nur ein Cloud UI bieten für alle Boxen bieten, muss es auch eine Möglichkeit geben, die Boxen auszuwählen, die gerade konfiguriert werden soll.
Jede Box könnte ein QR-Code erhalten, der beim Einloggen gescannt werden muss. Je nachdem kann diese Box dann im Account hinterlegt werden oder muss bei jedem Login gescannt werden.

\subsubsection{Boxdaten über Cloud UI ändern}
Dafür müsste eine API gebaut werden, damit der jeweilige Microcontroller der Box mit dem Cloud UI kommunizieren kann.

\subsubsection{Unterstützung von Youtube Music}
Eine Umsetzung könnte sich als umständlich erweisen, da es bisher noch keine dedizierte Youtube Music API gibt. In der bisherigen API gibt nur die Möglichkeit, nach Youtube Videos zu suchen \footnote{https://developers.google.com/youtube/v3/}. Dazu kommt auch das Problem, dass ein externes Wiedergabegerät wie bei Spotify Connect derzeitig in der Form noch nicht existiert.
\subsubsection{Unterstützung von Apple Music}
Ist möglich. Apple Music bietet hier mit deren MusicKit JS\footnote{https://developer.apple.com/documentation/musickitjs/} eine Möglichkeit hierfür.
\subsubsection{Unterstützung von Deezer}
Ist dank der Deezer API möglich. \footnote{http://developers.deezer.com/login?redirect=/api}
\subsubsection{Unterstützung von eigener Musik (USB Stick, MicroSD Karte, Cloud)}
Das nimmt vorerst zuviel Arbeit in Anspruch und ist daher momentan nicht geplant.

\subsection{Sonstiges}
\subsubsection{3D Print version}
Da unsere Box nicht übermäßig groß sein wird, sollten die meisten 3D-Drucker ausreichend sein. Das Modellieren einer 3D-Print Version ist am Ende abhängig von der Expertise der Gruppe. Abgesehen davon sollte es kein Problem darstellen.

\subsubsection{Sound Wiedergabe auf der Box selbst}
Manche Pi Modelle verfügen schon über ein Audio Anschluss, die Wiedergabe davon ist aber nicht die qualitativ beste. Daher wäre ein weiteres Audiomodul notwendig. Innerhalb der Raspberri Pi Reihe gibt es dafür Accessoires, die circa 20-30€ kosten. \footnote{https://www.raspberrypi.org/products/}.
Zur Wiedergabe auf der Box selbst müsste dafür eine Spotify Instanz laufen, damit auch diese als Connected Speaker erkannt wird.

\subsubsection{Box unter 30€ Kosten}
Mit einem Raspberri Pi wäre dieses Ziel möglich, es könnte aber kein Pi ab Model 3 verwendet werden, da diese über dem Ziel liegen.
Mit aber dem Raspberri Pi Zero W hätte man einerseits schon eingebautes Wlan und ein USB Port für den NFC Reader. Dieser ist für 10\$ erhältlich \footnote{https://www.raspberrypi.org/products/raspberry-pi-zero-w/}.

\section{Design Mockups}
\colorbox{red}{setze pdfs ein}

\section{Durchführung}
\subsubsection{Technologien und Hilfsmittel}
\colorbox{red}{Vue, vscode, devops krams, etc}
Entwickelt wird mit Visual Studio Code, da es eine einfache Nutzung des Linux-Subsystems ermöglicht. \footnote{https://code.visualstudio.com/docs/remote/wsl}.

\subsubsection{Projekt Management}
\colorbox{red}{ticket pool in github, alles zentral}

\subsubsection{Deployment Cycle}
\colorbox{red}{TODO: Umformulieren} ziehe ticket > assigne dich selbst > draft PR > wenn fertig, setzte \textit{undraft} > assigne 2 reviewer > merge master

\subsubsection{Probleme während der Durchführung}
\colorbox{red}{zb }
\section{Code Walkthrough}
\colorbox{red}{vielleicht interessante Code Snippets?}
\section{Testing}
\colorbox{red}{wie haben wir getestet, haben wir getestet?}
\section{Technische Diagramme}
\colorbox{red}{ER Diagramme, UML, solcher krams}
 \end{document}
