\documentclass[10pt, a4paper, draft]{article}
 \usepackage[ngerman]{babel}
 \usepackage[utf8]{inputenc}
 \usepackage[T1]{fontenc}

\title{Projektbreicht Smart Music Player}
\author{Anton Bracke\\Jan Eberlein\\Tom Calvin Haak\\Julian Hahn\\Nick Loewecke}

\begin{document}
\maketitle
\tableofcontents

\section{Einleitung}
\subsection{Projekt}
\subsection{Unternehmen}

\section{Durchführung}

\section{Technologien}

\section{Machbarkeitsstudie}

\subsection{NFC Tag}
\subsubsection{lesen}
\subsubsection{schreiben}

\subsection{Raspberri Pi}
\subsubsection{Docker Integration}
\subsubsection{Öffentlich zugängliches Web Interface}
\subsubsection{Zugriff auf NFC Reader}

\subsection{User Interface}
\subsubsection{Login via Spotify, Youtube, etc.}
\subsubsection{Gleichen Nutzer bei verschiedenen Loginvarianten wiedererkennen}
\subsubsection{Musik Artwork laden}
\subsubsection{Eigene Bilder hochladen}
\subsubsection{Spotify connect Lautsprecher auswählen}
\subsubsection{Spotify connect Lautsprecher speichern}
\subsubsection{In Cloud UI die jeweilige Box auswählen}
\subsubsection{Boxdaten über Cloud UI ändern}
\subsubsection{Unterstützung von Youtube}
\subsubsection{Unterstützung von Apple Music}
\subsubsection{Unterstützung von Deezer}
\subsubsection{Unterstützung von Tidal}
\subsubsection{Unterstützung von eigener Musik (USB Stick, MicroSD Karte, Cloud)}

\subsection{Sonstiges}
\subsubsection{3D Print version}
\subsubsection{Sound Wiedergabe auf der Box selbst}
\subsubsection{Box unter 30€ Kosten}

 \end{document}